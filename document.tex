%%This is a very basic article template.
%%There is just one section and two subsections.
\documentclass[a4paper,german,oneside,headsepline,footsepline,12pt]
{report}
\usepackage[ngerman]{babel}
\usepackage[latin1]{inputenc}
\usepackage[bottom]{footmisc}
\usepackage{parskip} 
\usepackage{setspace}
\onehalfspacing
\sloppy 
\setlength{\unitlength}{1cm}
\setlength{\oddsidemargin}{0.3cm}
\setlength{\evensidemargin}{0.3cm}
\setlength{\textwidth}{15.5cm}
\setlength{\topmargin}{-1.2cm}
\setlength{\textheight}{23cm}
\columnsep 0.5cm
\begin{document}
\pagestyle{plain}
\pagenumbering{arabic}
\setcounter{page}{1}
\tableofcontents
\newpage

\chapter*{Titel}
\addcontentsline{toc}{chapter}{Titel}
\section*{Motviation}
\addcontentsline{toc}{section}{Motivation}
Im Zuge meiner Diplomarbeit \glqq \textit{Die Evolution der repr�sentativen
Demokratie in der Informationsgesellschaft}\grqq ~befasste ich mich mit der
Rolle der Informations- und Kommunikationstechnologie (IKT) in der Informationsgesellschaft und den daraus resultierenden Auswirkungen auf die repr�sentative Demokratie. Diese thematische Auseinandersetzung war nicht nur herausfordernd, sondern half mir auch, eine Br�cke zwischen meinem Fachgebiet, der Informatik, und aktuellen gesellschaftspolitischen Themen zu schlagen. F�r mich ist es naheliegend, den eingeschlagenen Weg im Rahmen einer Dissertation zu vertiefen.

\section*{Problemstellung und Zielsetzung}
\addcontentsline{toc}{section}{Problemstellung und Zielsetzung}
Wo sehe ich bei der politischen Informationsverarbeitung, politischen
Kommunikation und politischen Partizipation demokratiepolitische Probleme?
Wie kann IKT helfen, diese Probleme so zu l�sen, dass eine Verst�rkung der
Demokratie eintritt?

Der Einsatz der IKT im politischen Kontext sowie dessen Einfluss auf die
vorherrschenden politischen Prozesse sind wichtige Faktoren des
politischen Geschehens. Die dadurch zunehmend wichtiger werdende Rolle der
IKT wirft neue Fragen im Hinblick auf die politische Informationsverarbeitung, politische
Kommunikation und politische Partizipation auf. (3ps) 

Probleme im Hinblick auf die 3ps
Bisherige Auswirkungen der IKT auf die 3ps
Quantifizierung der Demokratie
Informationsfluss ?!
Braucht die IKT Demokratie?
Wann ist ein Staat demokratisch?
Wie gestaltet sich die Wechselwirkung zwischen Demokratie und IKT. 
Kann man die Werbung als ein analoges Konzept betrachten?
F�r welche Demokratien sind IKT geeignet?
Was kann man konkret durch den Einsatz der IKT erreichen?
Probleme 1p: Erreichbarkeit, technische Barriere, 
Probleme 2p: 
Systemtheoretisch gesehen haben die IKT das Potenzial\ldots ?!
\section*{Methodisches Vorgehen}
\addcontentsline{toc}{section}{Methodisches Vorgehen}
Die methodische Vorgehensweise in dieser Arbeit gliedert sich wie folgt:
Zun�chst werden die politischen Prozesse formaltheoretisch kategorisiert.
Die Kategorisierung die ich als sinnvoll erachte unterteilt die politischen
Prozesse in politische Informationsverarbeitung, politische Kommunikation
und politische Partizipation. (vgl. Grunwald et al. 2006, S. 9ff., Dahl 2006,
S. 23ff, Hofkirchner 2010, S. 99ff.)
Todo.

\section*{Erwartetes Resultat}
\addcontentsline{toc}{section}{Erwartetes Resultat}
Das erwartete Resultat ist die Bew�ltigung und Beseitigung der eingangs
erw�hnten Problemstellung. Konkret soll dies mit Hilfe neuer IKT-spezifischer
und theoretischer L�sungen gelingen, welche die demokratischen Aspekte der drei
politischen Prozessarten verst�rken. Todo.

\section*{State-of-the-Art und Ausgangspunkt}
\addcontentsline{toc}{section}{State-of-the-Art und Ausgangspunkt}
Ausgehend von der vorgenommenen Kategorisierung wird die Wechselwirkung zwischen
Politik und IKT und die daraus ableitbaren Potenziale untersucht. Diese
Untersuchung findet auf mehreren Ebenen statt: 
Als erstes wird der bisherige Einsatz der IKT im politischen Kontext analysiert.
Das Erkennen und die Kennzeichnung politischer Prozesse stellt den
n�chsten Schritt dar. Zum Schluss wird noch untersucht, welchen Einfluss
die IKT bis dato auf die drei politische Prozessarten hatten. 
Diese vielschichtige Untersuchung erm�glicht mir, fundamentale Kenntnisse �ber
das Spannungsfeld Politik und IKT zu sammeln. 
Die Quantifizierung einer demokratischen Regierung 

Todo.
\end{document}
