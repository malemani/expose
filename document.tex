%%This is a very basic article template.
%%There is just one section and two subsections.
\documentclass[a4paper,german,oneside,headsepline,footsepline,12pt]
{report}
\usepackage[ngerman]{babel}
\usepackage[latin1]{inputenc}
\usepackage[bottom]{footmisc}
\usepackage{parskip} 
\usepackage{setspace}
\onehalfspacing
\sloppy 
\setlength{\unitlength}{1cm}
\setlength{\oddsidemargin}{0.3cm}
\setlength{\evensidemargin}{0.3cm}
\setlength{\textwidth}{15.5cm}
\setlength{\topmargin}{-1.2cm}
\setlength{\textheight}{23cm}
\columnsep 0.5cm
\begin{document}
\pagestyle{plain}
\pagenumbering{arabic}
\setcounter{page}{1}
\tableofcontents
\newpage

\chapter*{Titel}
\addcontentsline{toc}{chapter}{Titel}
\section*{Motviation}
\addcontentsline{toc}{section}{Motivation}
Im Zuge meiner Diplomarbeit \glqq \textit{Die Evolution der repr�sentativen
Demokratie in der Informationsgesellschaft}\grqq ~befasste ich mich mit der
Rolle der Informations- und Kommunikationstechnologie (IKT) in der Informationsgesellschaft und den daraus resultierenden Auswirkungen auf die repr�sentative Demokratie. Diese thematische Auseinandersetzung war nicht nur herausfordernd, sondern half mir auch, eine Br�cke zwischen meinem Fachgebiet, der Informatik, und aktuellen gesellschaftspolitischen Themen zu schlagen. F�r mich ist es naheliegend, den eingeschlagenen Weg im Rahmen einer Dissertation zu vertiefen.

\section*{Problemstellung und Zielsetzung}
\addcontentsline{toc}{section}{Problemstellung und Zielsetzung}
Der Einsatz der IKT im politischen Kontext sowie dessen Einfluss auf die
vorherrschenden politischen Prozesse sind wichtige Faktoren des
politischen Geschehens. Daraus leitet sich die zunehmend wichtiger
werdende Rolle der IKT ab, die politische Informationsverarbeitung, politische
Kommunikation und politische Partizipation

\section*{Methodisches Vorgehen}
\addcontentsline{toc}{section}{Methodisches Vorgehen}
Zun�chst ist es notwendig, die politischen Prozesse formaltheoretisch zu
kategorisieren. Um dieses Ziel zu erreichen, m�chte ich  
Todo.

\section*{Erwartetes Resultat}
\addcontentsline{toc}{section}{Erwartetes Resultat}
Todo.

\section*{State-of-the-Art und Ausgangspunkt}
\addcontentsline{toc}{section}{State-of-the-Art und Ausgangspunkt}
Todo.
\end{document}
